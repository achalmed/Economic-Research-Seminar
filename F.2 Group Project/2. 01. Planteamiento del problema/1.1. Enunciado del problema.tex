\subsection{Enunciado del problema}

La desigualdad económica es el distinto reparto de los ingresos, los activos o el bienestar entre el conjunto de habitantes, según explica la Organización para la Cooperación y el Desarrollo Económicos (OECD, 2018).\\
América Latina es la región más desigual de la región en términos socioeconómicos según Nora, L (2011) “la región latinoamericana es 19\% más desigual que el África subsahariana, 37 más desigual que el este asiático y 65\% más desigual que los países desarrollados”.  En efecto América Latina es, hoy, la región sin guerras más desigual del planeta: más que India, más que algunos países de África Subsahariana.\\
La región de América Latina y el Caribe es, según Oxfam, la más desigual en ingresos en el mundo, ya que en 2014 el 1 \% más rico poseía el 41 \% de la riqueza regional, mientras que el 99 \% restante debía repartirse el 60 \%.\\
La desigualdad socioeconómica impide luchar contra la pobreza, ampara que colectivos vulnerables vivan en condiciones de pobreza sin acceso a empleos dignos, servicios básicos, exponiéndose a una dieta pobre o sin una vivienda decente, se prolonga la exclusión y marginación social de personas y familias. \\
En este sentido es evidente la importancia de entender la pobreza.
Ravallion, G. (1991) afirma que la pobreza alude a niveles de vida.  Esto es: ¿cuántas personas no pueden satisfacer ciertas necesidades predeterminadas de consumo y acceso amplio a bienes públicos (servicios de salud, educación, vivienda)?  \\
Según INEI “La pobreza es una condición en la cual una o más personas tienen un nivel de bienestar inferior al mínimo socialmente aceptado.”\\
Las estadísticas indican que desde el año 2000 se ha conseguido reducir la tasa de incidencia de la pobreza en el Perú. \\
El siguiente cuadro del Banco Mundial presenta la tasa de incidencia de la pobreza, sobre la base de la línea de pobreza nacional (\% de la población) para el periodo 2004 – 2019.\\








Donde precisa que la tasa de incidencia de la pobreza entre 2004 y 2019 se redujo en 38.5\%. señala también que el principal periodo de reducción fue entre 2004 y 2011. \\ 
Entre el 2002 y 2016 la pobreza bajo de 54,30 a 20,7\% y la extrema pobreza de 24.2\% a 3.8\%, lo cual indica que en el gobierno de Alejandro Toledo la pobreza disminuyó en 5,2\% y la extrema pobreza en 10,4\%, en el gobierno de Alan García la pobreza se redujo en 21,4\% y la extrema pobreza en 7,5\%; y en el gobierno de Ollanta Humala la pobreza en 7,03\% y la extrema pobreza 2,54\%, llegando en el año 2018 al 20,50\% de pobreza y 2,8\% la extrema pobreza, con tendencia a seguir disminuyendo. \\
En nuestro país la pobreza pasó de 54\% (1990) a 20,50\% (2018) y la extrema pobreza de 24.2\% (1990) a 2.8\% (2018), disminuyendo significativamente en 33,5\% la pobreza y la extrema pobreza en 21.4\%, como consecuencia del crecimiento económico sostenido, el incremento del gasto social, la mejor calidad y focalización de los programas sociales, y el incremento de la inversión pública. \\
En el período 2001-2010, la pobreza decreció en 23,5 puntos porcentuales, al pasar de 54,8\% a 31,3\% en el 2010. \\
Donde precisa que la tasa de incidencia de la pobreza entre 2004 y 2019 se redujo en 38.5\%. señala también que el principal periodo de reducción fue entre 2004 y 2011. \\
Entre el 2002 y 2016 la pobreza bajo de 54,30 a 20,7\% y la extrema pobreza de 24.2\% a 3.8\%, lo cual indica que en el gobierno de Alejandro Toledo la pobreza disminuyó en 5,2\% y la extrema pobreza en 10,4\%, en el gobierno de Alan García la pobreza se redujo en 21,4\% y la extrema pobreza en 7,5\%; y en el gobierno de Ollanta Humala la pobreza en 7,03\% y la extrema pobreza 2,54\%, llegando en el año 2018 al 20,50\% de pobreza y 2,8\% la extrema pobreza, con tendencia a seguir disminuyendo. \\
En nuestro país la pobreza pasó de 54\% (1990) a 20,50\% (2018) y la extrema pobreza de 24.2\% (1990) a 2.8\% (2018), disminuyendo significativamente en 33,5\% la pobreza y la extrema pobreza en 21.4\%, como consecuencia del crecimiento económico sostenido, el incremento del gasto social, la mejor calidad y focalización de los programas sociales, y el incremento de la inversión pública.\\
En el período 2001-2010, la pobreza decreció en 23,5 puntos porcentuales, al pasar de 54,8\% a 31,3\% en el 2010.







En el año 2017, el 21,7\% de la población del país, que equivale en cifras absolutas a 6 millones 906 mil personas, se encontraban en situación de pobreza, es decir, tenían un nivel de gasto inferior al costo de la canasta básica de consumo compuesto por alimentos y no alimentos. Al comparar estos resultados con el nivel obtenido en el año 2016, se observa que la pobreza aumentó en 1,0 p.p, que equivale a 375 mil personas pobres, más que en el año 2016. \\






En el año 2019, el índice de pobreza monetaria afectó al 20,2\% de la población del país, con lo cual mantiene prácticamente los mismos niveles del año 2018; así lo dio a conocer el Instituto Nacional de Estadística e Informática (INEI) según los resultados de la Encuesta Nacional de Hogares (ENAHO) del año 2019. Asimismo, precisó que se considera población en condición de pobreza aquella cuyo gasto per cápita es inferior al valor de la Línea de Pobreza (LP), que es el equivalente monetario de una canasta básica de consumo alimentario y no alimentario. \\
Cuando se habla de pobreza también se hace referencia a la desigualdad social; un grupo social que es excluido ya que no cuenta con el mismo acceso a los recursos que otros grupos con poder si tienen, estas diferencias están marcadas con claridad entre las zonas urbanas y rurales. 
La aplicación errada de políticas públicas ha pronunciado aún más las diferencias, ya que no se ha permitido una integración de la multiculturalidad con la que cuenta el Perú, y muy por el contrario ha ocasionado un marcado distanciamiento entre ellos. \\
En un contexto de crecimiento económico sostenido, es necesario mejorar la formulación y aplicación de las políticas del país, en objetivos puntuales de desarrollo y asistencia social con el firme objetivo de hacerle frente a la pobreza y en un largo plazo mejorar los estándares de vida de su población en conjunto.




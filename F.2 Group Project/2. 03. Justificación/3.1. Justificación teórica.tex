\subsection{Justificación teórica}
	
La pobreza es un problema mundial que se ha venido combatiendo a lo largo de los años, cada país en forma individual ha implementado medidas y políticas con el único fin de reducir su nivel de pobreza o extrema pobreza. En caso particular de Perú, a pesar de los índices favorables en la reducción de la pobreza, en determinadas zonas se observa un margen significativo en la desigualdad socioeconómica el cual es paralelo al índice del nivel pobreza.\\
Es importante analizar el comportamiento de los índices de pobreza contrastado con indicadores de la desigualdad socioeconómica; para así poder establecer un grado de eficacia o trascendencia en el tiempo. Generar conocimiento en la relación que se establecen entre estos indicadores, de manera que se contraste con la teoría económica ya conocida o se generen nuevos aportes.


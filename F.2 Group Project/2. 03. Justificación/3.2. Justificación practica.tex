\subsection{Justificación practica}


Este trabajo tiene una justificación practica porque con él podrá ser posible su aplicación a la realidad para poder hacer frente a la desigualdad socioeconómica que predomina en distintas regiones en particular y sobre todo hacer frente a la pobreza que tanto sufrimiento ocasiona a una gran parte de la población peruana. \\
Los resultados de la presente investigación podrán aportar a ver mejor el panorama del país y con él se podrán diseñar estrategias de políticas públicas con los que en un mediano plazo podamos obtener resultados más eficaces en la reducción de la pobreza y la desigualdad socioeconómica en el Perú; además del gran aporte que significara para la toma de decisiones a nivel de gobierno.




\subsection{Sistema teórico}

  \subsubsection{Desigualdad Socio económica y la Pobreza}

    \paragraph{Desigualdad socio económica}
    La desigualdad socioeconómica es un problema de las sociedades contemporáneas, producto del desarrollo desigual de las diversas regiones del globo y de la imposición de ciertas ideologías o valoraciones de unos seres humanos por encima de otros. De hecho, la desigualdad socioeconómica es el origen de la discriminación, ya que esta última consiste en tratar de manera distinta a quienes se vean desfavorecidos económica, social o moralmente. 

    \paragraph{La Pobreza}
ONU, (2021) manifiesta que la pobreza va más allá de la falta de ingresos y recursos para garantizar unos medios de vida sostenibles. Es un problema de derechos humanos. Entre las distintas manifestaciones de la pobreza figuran el hambre, la malnutrición, la falta de una vivienda digna y el acceso limitado a otros servicios básicos como la educación o la salud. \\
Para el BID el concepto de pobreza se refiere a una privación de bienestar, siendo el bienestar un concepto bastante complejo y amplio, que en su nivel más básico suele abarcar aspectos como la alimentación, vestido, salud, y vivienda.
Mientras que Banco Mundial, (2018) La pobreza no implica únicamente una carencia de ingresos y de consumo: también se manifiesta en forma de niveles educativos bajos, resultados insatisfactorios en salud y nutrición, falta de acceso a servicios básicos y un entorno peligroso.
Según el método de INEI, (2007) define la pobreza como aquel conjunto de personas que no alcanzan a tener un nivel de satisfacción mínimo respecto a un conjunto de necesidades básicas relacionados con la salud, nutrición, educación, vivienda, etc. Es decir, parte de una conceptualización multidimensional de la pobreza al considerar los diferentes aspectos del desarrollo social.

    \paragraph{Medición de la pobreza}

  \subparagraph{Línea de pobreza}

  El método de la línea de la pobreza de INEI, (2007) utiliza una canasta de bienes y servicios (canasta normativa de satisfactores esenciales), cuyo valor per cápita (línea de pobreza) es equivalente al mínimo necesario para la sobrevivencia humana. Define a la población en situación de pobreza como aquel conjunto de personas cuyo nivel de bienestar, expresado en valor monetario, es inferior a la línea de pobreza.\\
Para la Dirección Provincial de Estadística de la provincia de Buenos Aires, (2010) el método más utilizado internacionalmente, a pesar de sus limitaciones es el método de la Línea de Pobreza (LP), el cual utiliza el ingreso o el gasto de consumo como medidas del bienestar, estableciéndose un valor per cápita de una canasta mínima de consumo necesario para la sobrevivencia, es decir, una canasta de satisfactores esenciales, el cual permite la diferenciación de los niveles de pobreza.

  \subparagraph{Las necesidades básicas insatisfechas (NBI)}

Este método INEI, (2007) define la pobreza como aquel conjunto de personas que no alcanzan a tener un nivel de satisfacción mínimo respecto a un conjunto de necesidades básicas relacionados con la salud, nutrición, educación, vivienda, etc. Es decir, parte de una conceptualización multidimensional de la pobreza al considerar los diferentes aspectos del desarrollo social. \\
DPE, (2010) El método de medición de las Necesidades Básicas Insatisfechas (NBI) toma en consideración un conjunto de indicadores relacionados con necesidades básicas estructurales (vivienda, educación, salud, infraestructura pública, etc.) que se requiere para evaluar el bienestar individual.

  \subparagraph{Método integrado}

El método integrado de INEI, (2007) de medición de la pobreza no es más que la combinación de ambos métodos y es utilizado fundamentalmente con el propósito de reconocer segmentos diferenciados entre los pobres, y también para entender el énfasis que debe ponerse en las políticas antipobreza.\\
DPE, (2010) El tercer método, denominado Método Integrado de medición de la pobreza, combina los métodos de la línea de pobreza y necesidades básicas insatisfechas. Con este método se clasifica a la población en los siguientes cuatro grupos: 

\begin{itemize}
\item Pobres crónicos que son los grupos más vulnerables porque tienen al menos una NBI e ingresos o gastos por debajo de la línea de pobreza. 
\item Pobres recientes, es decir, aquellos que tienen sus necesidades básicas satisfechas pero que sus ingresos están por debajo de la línea de pobreza. 
\item Pobres inerciales, que son aquellos que tienen al menos una necesidad básica insatisfecha, pero sus ingresos o gastos están por encima de la línea de pobreza.
\item Integrados socialmente, es decir los que no tienen necesidades básicas insatisfechas y sus gastos están por arriba de la línea de pobreza.

\end{itemize}

  \subsubsection{Empleo e Índice de Desarrollo Humano (IDH)}
La Organización Internacional del Trabajo define el pleno empleo como el escenario donde hay trabajo para todos los que quieren trabajar y están en busco de ello. \\
Existen dos tipos de empleo; formal e informal (Serie de Estudios Económicos, 2015).\\
En el empleo formal se encuentran las personas cuyo trabajo y derechos laborales son reconocidos; mientras que el empleo informal es lo contrario, puesto que aunque los trabajadores perciben un pago en remuneracion a su trabajo, estos no reciben el mismo reconocimiento ni los derechos laborales.\\
Para Ricardo (1959), el principal determinante del empleo era la acumulación del capital, el cual significa el uso excedente para contratar trabajadores asalariados.\\
El Programa de las Naciones Unidas (2021), ha venido publicando desde hace tres décadas, el Informe sobre Desarrollo Humano, el centra su análisis en el comportamiento del desarrollo a nivel mundial. El IDH es un indicador más confiable que el crecimiento del PBI, pues este último basa su razón solo en el comportamiento del nivel de ingreso, mientras que el IDH considera también otras dimensiones más complejas como nivel de ingreso per cápita, esperanza de vida y el acceso a la educación. \\
El PDNU (2021) para una interpretación efectiva del IDH, considera cuatro categorías: 

\begin{itemize}
\item Desarrollo humano muy elevado (mayores a 0.8)
\item Desarrollo humano elevado (entre 0.7 y 0.7999)
\item Desarrollo humano medio (entre 0.55 y 0.6999)
\item Desarrollo humano bajo (menores a 0.55)

\end{itemize}

  \subsubsection{Índice de Theilen y grado de pobreza}

    \paragraph{Índice de Theilen}
Wikipedia, (2021) El índice de Theilen es una estadística que se utiliza principalmente para medir la desigualdad y otros fenómenos económicos, aunque también se ha utilizado para medir la segregación racial. \\
Según, Casatañeda, (2013) Esta medición, aunque es arbitraria y es una aplicación sacada directamente de la física y la teoría de la información para ser llevada a la economía. Es muy bien aceptada entre los que estudian la desigualdad y durante mucho tiempo ha sido una medición alternativa al coeficiente Gini.

    \paragraph{Grado de pobreza}

  \subparagraph{Extremo pobre}

Para Damm Arnal, (2017) comprende a las personas cuyos hogares tienen ingresos o consumos per cápita inferiores al valor de una canasta mínima de alimentos.

\subparagraph{Pobre}

Según Damm Arnal, (2017) una persona se encuentra en situación de pobreza si tiene al menos una carencia social, de las seis enumeradas en el primer párrafo, y su ingreso es insuficiente para adquirir los bienes y servicios que componen la canasta básica alimentaria y no alimentaria.

  \subparagraph{No pobre}

Se considera no pobres a las personas cuyo ingreso o consumo es suficiente para mantener un nivel de vida elevado o de calidad.

  \subsubsection{Desempleo e Índice de Gini}

    \paragraph{Desempleo}

Según De Gregorio, (2007) El desempleo es un indicador importante para medir el desempeño de una economía en términos de actividad, además que siempre es aquella fracción de los que quieren trabajar, pero no consiguen hacerlo. \\
Para. Pissarides, (1990) existe desempleo a causa de fricciones en el mercado laboral en el proceso de búsqueda de empleo por parte de los trabajadores y de contratación por parte de las firmas.\\
Modigliani, Fitoussi, Moro, Snower, \& Solow (1999), señalan que el desempleo afecta principalmente a los trabajadores poco calificados pues cuando los puestos de trabajo disponibles se reducen los trabajadores más calificados desplazan calificados mientras que cuando los puestos de trabajo aumentan los trabajadores calificados obtienen mejores empleos dejando libres los puestos de trabajo que ocupaban a los trabajadores poco calificados.

    \paragraph{Clases de desempleo}

  \subparagraph{Desempleo friccional}
Es cuando se tiene libertad para escoger los empleos siempre existen personas que se traslada de un trabajo a otro.

  \subparagraph{Desempleo estructural}
Ocurre cuando con el tiempo se presenta cambios en la estructura de la demanda del consumidor y del avance tecnológico los mismos que alteran la estructura de la demanda global de trabajo.

  \subparagraph{Desempleo cíclico}
Este tipo de desempleo es causado por la fase recesiva del ciclo económico cuando quiebran las empresas con escasa solvencia económico y dejan sin trabajo a muchas personas.

    \paragraph{Índice de Gini}

Para Montero Castellanos, (2021) el coeficiente de Gini es una de las métricas utilizada para orientarnos respecto a la desigualdad económica. Cuanto mayor es el índice de Gini, mayor es la desigualdad de los ingresos en la población. 
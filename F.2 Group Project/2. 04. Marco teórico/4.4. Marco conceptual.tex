\subsection{Marco conceptual}

  \subsubsection{Desigualdad socioeconómica}
  
  La desigualdad socioeconómica es un problema actual, el cual se origina como resultado a un desarrollo desigual, creando a su vez un concepto falso de superioridad.
  
  \subsubsection{La Pobreza}
  
  Según el método de INEI, (2007) define la pobreza como aquel conjunto de personas que no alcanzan a tener un nivel de satisfacción mínimo respecto a un conjunto de necesidades básicas relacionados con la salud, nutrición, educación, vivienda, etc. Es decir, parte de una conceptualización multidimensional de la pobreza al considerar los diferentes aspectos del desarrollo social.
  
  \subsubsection{Empleo}
  
  El empleo se define como aquel escenario donde existe trabajo para todas las personas que quieran trabajar o estén buscándolo. Existen dos tipos de empleo: formal (reconocido y con todos los derechos laborales) e informal (no está reconocido y no goza de los derechos laborales).
  
  \subsubsection{Índice de Desarrollo Humano (IDH)}
  
  Es aquel indicador que basa su estudio en dimensiones más complejas, lo cual permite obtener un resultado más confiable; mide el nivel de desarrollo de cada país, y sus principales variables son: el nivel de ingreso per cápita, esperanza de vida y el acceso a la educación. Es aquel indicador que basa su estudio en dimensiones más complejas, lo cual permite obtener un resultado más confiable; mide el nivel de desarrollo de cada país, y sus principales niveles variables son: el nivel de ingreso per cápita, esperanza de vida y el acceso a la educación. 
  
  \subsubsection{Índice de Theilen}
  
  El índice de Theilen es una estadística que se utiliza principalmente para medir la desigualdad y otros fenómenos económicos, aunque también se ha utilizado para medir la segregación racial. \\
Cuando el índice de Theilen es cercano a cero la sociedad es igual mientras el índice se va al infinito la es sociedad es desigual. 

  \subsubsection{Grado de pobreza}
  
\paragraph{Extremo pobre} 
Para Damm Arnal, (2017) comprende a las personas cuyos hogares tienen ingresos o consumos per cápita inferiores al valor de una canasta mínima de alimentos.

\paragraph{Pobre}

Según Damm Arnal, (2017) una persona se encuentra en situación de pobreza si tiene al menos una carencia social, de las seis enumeradas en el primer párrafo, y su ingreso es insuficiente para adquirir los bienes y servicios que componen la canasta básica alimentaria y no alimentaria.


  \subsubsection{Desempleo}
  
  Modigliani, Fitoussi, Moro, Snower, \& Solow (1999), señalan que el desempleo afecta principalmente a los trabajadores poco calificados pues cuando los puestos de trabajo disponibles se reducen los trabajadores más calificados desplazan calificados mientras que cuando los puestos de trabajo aumentan los trabajadores calificados obtienen mejores empleos dejando libres los puestos de trabajo que ocupaban a los trabajadores poco calificados.
  
  \subsubsection{Índice de Gini}
  
  El coeficiente Gini es un indicador que nos permite medir la desigualdad de los ingresos de la población.
  
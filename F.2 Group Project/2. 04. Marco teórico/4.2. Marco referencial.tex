\subsection{Marco referencial}
	
La desigualad socioeconómica es un tema de poca frecuencia y permanente interés. 
Según Cotler (2011) en los años de 1960 la forma de pensar sobre la desigualdad socioeconómica estaba influenciado por la sociología funcionalista - estructuralista, así como el Marxismo. \\
Así, puede afirmarse que, en las ciencias sociales peruanas, han sido muy influyentes, desde la década de 1960, dos aproximaciones con pretensiones de “gran teoría”: el funcionalismo estructural y el marxismo. \\
Según Jimenez, (2016) la desigualdad es un problema porque afecta a la calidad de vida de las personas y restringe las capacidades y libertades individuales.\\
En este sentido la desigualdad es una externalidad negativa y los individuos no están dispuestos a tolerar un grado elevado de desigualdad. Figueroa, (2001)\\
En cuanto a la pobreza, en términos monetarios, alude a la falta de ingresos para poder cubrir el costo de una canasta básica de consumo; por otro lado, la carencia de ingresos suficientes "está asociada a la carencia del capital humano necesario para acceder a ciertos empleos", o a la falta de "capital financiero, tierra y conocimientos gerenciales y tecnológicos para desarrollar una actividad empresarial" (CEPAL, 2000) \\
Si hacemos referencia a un enfoque más complejo, analizamos al premio Noble Amartya Sen, quien menciona: "la pobreza debe concebirse como la privación de capacidades básicas y no meramente como la falta de ingresos, que es el criterio habitual con el que se identifica la pobreza" (Sen, 2000:114).  Además, el autor afirma que "cuanto mayor sea la cobertura de la educación básica y de la asistencia sanitaria, más probable es que incluso las personas potencialmente pobres tengan más oportunidades de vencer la miseria" (Sen, 2000).\\
Otro enfoque para considerar es el de la ya conocida pobreza humana; este enfoque propuesto por el Programa de las Naciones Unidad para el Desarrollo (PNUD) afirma que "el concepto de pobreza humana considera que la falta de ingreso suficiente es un factor importante de privación humana, pero no el único"; por lo tanto, se debe tener en cuenta que no todo empobrecimiento puede relacionarse únicamente con el ingreso. ""Si el ingreso no es la suma total de la vida humana, la falta de ingreso no puede ser la suma total de la privación humana" (PNUD, 2000)."\\
Para Mendoza y García (2006), la pobreza puede tener cambios más significativos manteniéndose un crecimiento económico, pero también es imprescindible el accionar del Estado en la promoción de equidad de oportunidades de desarrollo de la persona (inversión en capital humano) el cual, en un largo plazo, mediante el incremento de la productividad, favorecerá también al crecimiento económico el cual se hará sostenible de forma automática.\\
Boragina (2006) en su artículo, afirma que existen tres falacias económicas las cuales en la economía moderna quedan desfasadas absolutamente, él afirma que: la riqueza es dinámica, la producción y distribución son un único fenómeno y que el valor es el que genera trabajo; de esta manera, señala que al aferrarnos a las teorías de la economía antigua estamos perpetuando la pobreza en el mundo.







\subsection{Marco histórico}
	
\subsubsection{Enfoque Subjetivo}

Según De La Piedra, (1984) la pobreza desde el punto de vista subjetivo significa considerar que cualquier persona o familia puede dar su juicio acerca del grado en que satisface sus necesidades básicas o con otras palabras sobre el grado al cual ella misma piensa que sus medios les sirven para alcanzar sus fines. Dependiendo de ese juicio se le considera pobre o no pobre.

\subsubsection{Enfoque Relativo}
Desde el enfoque de la privación relativa de Townsend, (1962) se concibe un umbral del ingreso, de acuerdo con el tamaño y el tipo de familia, por debajo del cual el abandono o la exclusión de la membresía activa de la sociedad se acentúa en forma desproporcionada. La existencia de ese umbral depende de la evidencia científica que pueda recopilarse.
Para Murillo Alfaro, (1993) En el enfoque de pobreza relativa, el bienestar de un individuo o familia no depende de su nivel absoluto de consumo o gasto, sino del obtenido en relación con otros miembros de la sociedad. El punto de partida consiste en buscar un referente que puede ser el promedio o un grupo sociales determinado. De este modo, se define la pobreza como una situación de insatisfacción de necesidades básicas en relación con el referente social.

\subsubsection{Enfoque absoluto}

Sen, (1981) Señala que hay un núcleo irreductible de privación absoluta en la idea de pobreza, que se traduce en manifestaciones de muerte por hambre, desnutrición y penuria visible en un diagnóstico de la pobreza, sin tener que indagar primero en un panorama relativo. Consecuentemente, la idea de pobreza relativa complementa y no suplanta el enfoque absolutista de la pobreza.
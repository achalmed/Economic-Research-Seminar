\section{\large MARCO TEÓRICO}

\subsection{Marco histórico}
	
\subsubsection{Enfoque Subjetivo}

De La Piedra (1984), afirma que la pobreza subjetiva “significa considerar que cualquier individuo o familia puede opinar sobre qué tan bien se están satisfaciendo sus necesidades básicas, en otras palabras, sobre el grado al cual ella misma piensa que sus medios les sirven para alcanzar sus fines”. Por tanto, de acuerdo a este juicio se le considera pobre o no pobre.

\subsubsection{Enfoque Relativo}
Según Townsend (1962), en su enfoque de la privación relativa, “se concibe un umbral del ingreso, de acuerdo con el tamaño y el tipo de familia, por debajo del cual el abandono o la exclusión de la membresía activa de la sociedad se acentúa en forma desproporcionada”. 

Para Murillo Alfaro (1993), “en el enfoque de pobreza relativa, la felicidad de un individuo o de una familia no depende de su nivel absoluto de consumo o gasto, sino de su nivel de logro en las relaciones con otros miembros de la sociedad”. El punto de partida es buscar un punto de referencia que pueda ser una media o un grupo social específico. Así, la pobreza se define como un estado de necesidades básicas insatisfactorias en relación con la sociedad.

\subsubsection{Enfoque absoluto}
Sen (1981), enfatiza que “existe un núcleo innegable de privación absoluta en la idea de pobreza, que conduce a manifestaciones visibles de muerte por hambre, desnutrición y penuria en un diagnóstico de la pobreza, sin tener que indagar primero en un panorama relativo”. Por lo cual, la idea de pobreza relativa integra y no reemplaza el enfoque de pobreza autoritaria.


\subsection{Marco referencial}
	
La desigualad socioeconómica es un tema de poca frecuencia y permanente interés. Según Cotler (2011), en los años de 1960 la forma de pensar sobre la desigualdad socioeconómica estaba influenciado por la sociología funcionalista - estructuralista, así como el Marxismo. 

Así, se puede decir que, en las ciencias sociales peruanas, dos enfoques que pretenden ser "gran teoría" han sido muy influyentes desde la década de 1960: el funcionalismo estructural y el marxismo.

Según Jiménez (2016), la desigualdad es un problema porque afecta a la calidad de vida de las personas y restringe las capacidades y libertades individuales.

Por tanto, la desigualdad se considera como una externalidad negativa donde las personas o familias no están dispuestos a tolerar altos niveles de desigualdad. Figueroa, (2001).

En cuanto a la pobreza, en términos monetarios, alude a la falta de ingresos para poder cubrir el costo de una canasta básica de consumo; por otro lado, la carencia de ingresos suficientes "está asociada a la carencia del capital humano necesario para acceder a ciertos empleos", o a la falta de "capital financiero, tierra y conocimientos gerenciales y tecnológicos para desarrollar una actividad empresarial", CEPAL (2000).

Si hacemos referencia a un enfoque más complejo, analizamos al premio Noble Amartya Sen, quien menciona: "la pobreza debe concebirse como la privación de capacidades básicas y no meramente como la falta de ingresos, que es el criterio habitual con el que se identifica la pobreza" (Sen, 2000).  Además, el autor afirma que "cuanto mayor sea la cobertura de la educación básica y de la asistencia sanitaria, más probable es que incluso las personas potencialmente pobres tengan más oportunidades de vencer la miseria" (Sen, 2000).

Otro enfoque para considerar es el de la ya conocida pobreza humana; este enfoque propuesto por el Programa de las Naciones Unidad para el Desarrollo (PNUD) afirma que "el concepto de pobreza humana considera que la falta de ingreso suficiente es un factor importante de privación humana, pero no el único"; por lo tanto, se debe tener en cuenta que no todo empobrecimiento puede relacionarse únicamente con el ingreso. "Si el ingreso no es la suma total de la vida humana, la falta de ingreso no puede ser la suma total de la privación humana" PNUD (2000).

Para Mendoza y García (2006), la pobreza puede tener cambios más significativos manteniéndose un crecimiento económico, pero también es imprescindible el accionar del Estado en la promoción de equidad de oportunidades de desarrollo de la persona (inversión en capital humano) el cual, en un largo plazo, mediante el incremento de la productividad, favorecerá también al crecimiento económico el cual se hará sostenible de forma automática.

Boragina (2006), en su artículo, afirma que existen tres falacias económicas las cuales en la economía moderna quedan desfasadas absolutamente, él afirma que: la riqueza es dinámica, la producción y distribución son un único fenómeno y que el valor es el que genera trabajo; de esta manera, señala que al aferrarnos a las teorías de la economía antigua estamos perpetuando la pobreza en el mundo.


\subsection{Sistema teórico}

  \subsubsection{Desigualdad Socio económica y la Pobreza}

    \paragraph{Desigualdad socio económica}
La desigualdad socioeconómica es un problema actual, producto del desarrollo desigual entre las diferentes regiones del mundo y la imposición de ciertas ideologías o definiciones, el precio de unas personas en relación con otras. De hecho, la desigualdad socioeconómica está en la raíz de la discriminación, ya que la desigualdad incluye un trato diferente de quienes se encuentran en desventaja económica, social o moral.

    \paragraph{La Pobreza}
Las Naciones Unidas (2021), afirma que “la pobreza va más allá de la falta de ingresos y recursos para garantizar medios de vida sostenibles. Es una cuestión de derechos humanos”. Entre las diferentes manifestaciones de la pobreza se encuentran: el hambre, la desnutrición, la falta de una vivienda digna y el acceso limitado a otros servicios básicos como la salud o la educación.

Para el BID “la pobreza es la privación de bienestar, entendiendo el bienestar un concepto bastante complejo y amplio, y su nivel más básico abarca aspectos como la alimentación, vestido, salud, y vivienda”

Mientras para el Banco Mundial (2018), “la pobreza no solo implica falta de ingresos y consumo: también se manifiesta por bajos niveles de educación, resultados insatisfactorios en salud y nutrición, falta de acceso a servicios básicos.” 

Según INEI, (2007) “la pobreza se define como un grupo de personas que no alcanzan un nivel mínimo de satisfacción para una serie de necesidades básicas relacionadas con la salud, la nutrición, la educación, la vivienda y otros.”


    \paragraph{Medición de la pobreza}

  \subparagraph{Línea de pobreza}

El método de la línea de pobreza del INEI (2007), usa una canasta de bienes y servicios (la canasta estándar de necesidades), cuyo valor per cápita (la línea de pobreza) es equivalente al mínimo necesario para la subsistencia. 

Para la Dirección Provincial de Estadística de la provincia de Buenos Aires (2010), “el método más utilizado en el mundo, a pesar de sus limitaciones, es el Método de la Línea de Pobreza (PL), que utiliza el ingreso o el gasto de consumo como medida de bienestar”, estableciendo así, un valor per cápita de una canasta mínima de consumo necesaria para la supervivencia, es decir, una canasta de factores esenciales de satisfacción, que distingue grados de pobreza.


  \subparagraph{Las necesidades básicas insatisfechas (NBI)}

Este método, INEI (2007), define la pobreza como un grupo de personas que no cumplen con el nivel mínimo de satisfacción para un rango de necesidades básicas relacionadas con la educación, vivienda, salud, nutrición, etc. Es decir, parte de un concepto multidimensional de pobreza considerando distintos aspectos del desarrollo social.

Según el DPE (2010), “la medida de las necesidades básicas insatisfechas (NBI) tiene en cuenta un conjunto de indicadores relacionados con las necesidades estructurales básicas (vivienda, educación, salud, infraestructura pública, etc.) que se necesitan para evaluar el nivel de felicidad de un individuo.”


  \subparagraph{Método integrado}

La Medida Integrada de Pobreza del INEI (2007), no es más que una combinación de los dos métodos y se utiliza esencialmente con la finalidad de identificar a los distintos segmentos de los pobres, entendiendo que se debe poner énfasis en las políticas de reducción de la pobreza. 

DPE (2010), El tercer método, denominado medida de pobreza integrada, combina métodos de línea de pobreza y necesidades básicas insatisfechas. Con este método, la población se fragmenta en cuatro grupos específicos:

\begin{itemize}
\item Los crónicamente pobres son el grupo más vulnerable porque tienen al menos una NBI e ingresos o gastos por debajo de la línea de pobreza. 

\item Los pobres recientes, son aquellas personas cuyas necesidades básicas están satisfechas pero sus ingresos están por debajo de la línea de pobreza. 

\item Pobres por inercia, son los que tienen como mínimo una necesidad básica insatisfecha, pero sus ingresos o gastos están por encima de la línea de pobreza. 

\item Inclusión social, son aquellos que no tienen necesidades básicas insatisfechas y su gasto está por encima de la línea de pobreza.

\end{itemize}

  \subsubsection{Empleo e Índice de Desarrollo Humano (IDH)}
El concepto de pleno empleo de la Organización Internacional del Trabajo (OIT) es el escenario de empleo para todos los que quieren trabajar y lo buscan.

Existen dos tipos de empleo; formal e informal (Serie de Estudios Económicos, 2015). En el empleo formal se encuentran las personas cuyo trabajo y derechos laborales son reconocidos; mientras que el empleo informal es lo contrario, puesto que aunque los trabajadores perciben un pago en remuneracion a su trabajo, estos no reciben el mismo reconocimiento ni los derechos laborales.

Para Ricardo (1959), el principal determinante del empleo era la acumulación del capital, el cual significa el uso excedente para contratar trabajadores asalariados.

El Programa de las Naciones Unidas (2021), ha venido publicando desde hace tres décadas, el Informe sobre Desarrollo Humano, el centra su análisis en el comportamiento del desarrollo a nivel mundial. El IDH es un indicador más confiable que el crecimiento del PBI, pues este último basa su razón solo en el comportamiento del nivel de ingreso, mientras que el IDH considera también otras dimensiones más complejas como nivel de ingreso per cápita, esperanza de vida y el acceso a la educación. 

El PDNU (2021), para una interpretación efectiva del IDH, considera cuatro categorías: 

\begin{itemize}
\item Desarrollo humano muy elevado (mayores a 0.8)
\item Desarrollo humano elevado (entre 0.7 y 0.7999)
\item Desarrollo humano medio (entre 0.55 y 0.6999)
\item Desarrollo humano bajo (menores a 0.55)

\end{itemize}

  \subsubsection{Índice de Theilen y grado de pobreza}

    \paragraph{Índice de Theilen}
Wikipedia, (2021) “El índice de Theilen es una estadística que se utiliza principalmente para medir la desigualdad y otros fenómenos económicos, aunque también se ha utilizado para medir la segregación racial.”
 
Según Castañeda (2013), Esta medida es una aplicación prestada directamente de la física y la teoría de la información, que está bien aceptado entre los investigadores de la desigualdad y durante mucho tiempo ha sido una medida alternativa del coeficiente de Gini.

    \paragraph{Grado de pobreza}

  \subparagraph{Extremo pobre}

Para Damm Arnal (2017), “comprende a las personas cuyos hogares tienen ingresos o consumos per cápita inferiores al valor de una canasta mínima de alimentos.”

\subparagraph{Pobre}

Damm Arnal (2017), define a pobre a “la persona que se encuentra en situación de pobreza si padece al menos una privación social, sus ingresos no son suficientes para adquirir los bienes y servicios que constituyen las canastas básicas alimentarias y no alimentarias.”

  \subparagraph{No pobre}

Se considera no pobres a aquellas personas cuyo ingreso o consumo es suficiente para mantener un nivel de vida elevado o de calidad.

  \subsubsection{Desempleo e Índice de Gini}

    \paragraph{Desempleo}

Según De Gregorio (2007), El desempleo es un indicador importante para medir el desempeño de una economía en términos de actividad, además que siempre es aquella fracción de los que quieren trabajar, pero no consiguen hacerlo.

Para. Pissarides (1990), “existe desempleo a causa de fricciones en el mercado laboral en el proceso de búsqueda de empleo por parte de los trabajadores y de contratación por parte de las firmas.”

\cite{Modigliani1998}, afirman que el desempleo afecta principal y directamente a los trabajadores poco calificados, ya que a medida que disminuye la disponibilidad de empleo, los trabajadores más calificados desplazan a los trabajadores poco calificados; mientras que, si aumenta el empleo ocurre todo lo contario, es decir, los trabajadores calificados obtienen mejores trabajos.


    \paragraph{Clases de desempleo}

  \subparagraph{Desempleo friccional}
Es cuando se tiene libertad para escoger los empleos siempre existen personas que se traslada de un trabajo a otro.

  \subparagraph{Desempleo estructural}
Ocurre cuando con el tiempo se presenta cambios en la estructura de la demanda del consumidor y del avance tecnológico los mismos que alteran la estructura de la demanda global de trabajo.

  \subparagraph{Desempleo cíclico}
Este tipo de desempleo es causado por la fase recesiva del ciclo económico cuando quiebran las empresas con escasa solvencia económico y dejan sin trabajo a muchas personas.

    \paragraph{Índice de Gini}

Para Montero Castellanos (2021), “el coeficiente de Gini es una de las métricas utilizada para orientarnos respecto a la desigualdad económica. Cuanto mayor es el índice de Gini, mayor es la desigualdad de los ingresos en la población.” 

\subsection{Marco conceptual}

  \subsubsection{Desigualdad socioeconómica}
  
  La desigualdad socioeconómica es un problema actual, el cual se origina como resultado a un desarrollo desigual, creando a su vez un concepto falso de superioridad.
  
  \subsubsection{La Pobreza}
  
  Según el método de INEI (2007), se define a la pobreza como un grupo de personas que no cumplen con el nivel mínimo de satisfacción para un rango de necesidades básicas relacionadas con la salud, nutrición, educación, vivienda, etc. Es decir, parte de un concepto multidimensional de pobreza considerando diferentes aspectos del desarrollo social.
  
  \subsubsection{Empleo}
  
  El empleo se define como aquel escenario donde existe trabajo para todas las personas que quieran trabajar o estén buscándolo. Existen dos tipos de empleo: formal (reconocido y con todos los derechos laborales) e informal (no está reconocido y no goza de los derechos laborales).
  
  \subsubsection{Índice de Desarrollo Humano (IDH)}
  
  Es aquel indicador que basa su estudio en dimensiones más complejas, lo cual permite obtener un resultado más confiable; mide el nivel de desarrollo de cada país, y sus principales variables son: el nivel de ingreso per cápita, esperanza de vida y el acceso a la educación. Es aquel indicador que basa su estudio en dimensiones más complejas, lo cual permite obtener un resultado más confiable; mide el nivel de desarrollo de cada país, y sus principales niveles variables son: el nivel de ingreso per cápita, esperanza de vida y el acceso a la educación. 
  
  \subsubsection{Índice de Theilen}
  
  El índice de Theilen es una estadística que se utiliza principalmente para medir la desigualdad y otros fenómenos económicos.
  
Cuando el índice de Theilen es cercano a cero la sociedad es igual mientras el índice se va al infinito la es sociedad es desigual. 


  \subsubsection{Grado de pobreza}
  
\paragraph{Extremo pobre} 
Para Damm Arnal (2017), incluye a las personas cuyos hogares tienen un ingreso per cápita o un nivel de consumo por debajo del valor de una canasta mínima de alimentos.

\paragraph{Pobre}

Según Damm Arnal (2017), una persona se encuentra en situación de pobreza si tiene al menos una carencia social (de las seis enumeradas en el primer párrafo) y su ingreso es insuficiente para adquirir los bienes y servicios que componen la canasta básica alimentaria y no alimentaria.


  \subsubsection{Desempleo}
  
  Modigliani, Fitoussi, Moro, Snower, y Solow (1999), afirman que el desempleo afecta principal y directamente a los trabajadores poco calificados, ya que a medida que disminuye la disponibilidad de empleo, los trabajadores más calificados desplazan a los trabajadores poco calificados; mientras que, si aumenta el empleo ocurre todo lo contario, es decir, los trabajadores calificados obtienen mejores trabajos.
  
  \subsubsection{Índice de Gini}
  
  El coeficiente Gini es un indicador que nos permite medir la desigualdad de los ingresos de la población.
  		
